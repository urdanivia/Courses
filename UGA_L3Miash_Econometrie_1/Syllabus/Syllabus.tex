%\documentclass[12pt, reqno]{amsart}
%\documentclass[12pt, reqno, fleqn]{amsart}
\documentclass[12pt, reqno]{amsart}
\usepackage[svgnames]{xcolor}
\usepackage{amsmath}
\usepackage{amsthm}
\usepackage{amsfonts}
\usepackage{amssymb}
\usepackage{graphicx}
\usepackage{hyperref}
\usepackage[left=1in,right=1in,top=0.9in,bottom=0.9in]{geometry}
\usepackage{multirow}
\usepackage{verbatim}
%\usepackage[greek,frenchb]{babel}
\usepackage[utf8]{inputenc}
\usepackage[T1]{fontenc}
\usepackage{pdfpages}
\usepackage{enumitem}
\usepackage{natbib}
\usepackage{ctable}
\usepackage{lscape}
\usepackage{array}
\usepackage{bbm}
\usepackage{hyperref}
\usepackage{graphicx}
\usepackage{epsf}
\usepackage{fancyhdr}
\usepackage{setspace}
\usepackage{verbatim}
\usepackage{courier}
\usepackage{color}
\usepackage[normalem]{ulem}
\usepackage{hyperref}
\usepackage{fncylab}
\usepackage{authblk}
\usepackage{textcomp}
%\usepackage{sectsty}
\usepackage{lipsum}
\hypersetup{
  colorlinks   = true, %Colours links instead of ugly boxes
  urlcolor     = blue, %Colour for external hyperlinks
  linkcolor    = DarkBlue, %Colour of internal links
  citecolor   = DarkBlue %Colour of citations
}
\newtheorem{theoreme}{Théorème}
\newtheorem{conjecture}{Conjecture}[section]
\newtheorem{corollary}{Corollary}[section]
\newtheorem{lemme}{Lemme}[section]
\newtheorem{proposition}{Proposition}[section]
\newtheorem{assumption}{}
\labelformat{assumption}{H\arabic{assumption}}
\renewcommand{\theassumption}{Hypothèse H\arabic{assumption}}
\newtheorem{remarque}{Remarque}
\newtheorem{definition}{}
\labelformat{definition}{D\arabic{definition}}
\newtheorem{exemple}{Exemple}
\renewcommand{\thedefinition}{Définition D\arabic{definition}}
\newcommand{\argmax}{\operatornamewithlimits{arg\,max\,}}
\newcommand{\argmin}{\operatornamewithlimits{arg\,min\,}}
\providecommand{\plim}{\operatornamewithlimits{plim}}
\def\inprobLOW{\rightarrow_p}
\def\inprobHIGH{\,{\buildrel p \over \rightarrow}\,} 
\def\inprob{\,{\inprobHIGH}\,} 
\def\indist{\,{\buildrel d \over \rightarrow}\,} 
\def\F{\mathbb{F}}
\def\R{\mathbb{R}}
\newcommand{\gmatrix}[1]{\begin{pmatrix} {#1}_{11} & \cdots &
    {#1}_{1n} \\ \vdots & \ddots & \vdots \\ {#1}_{m1} & \cdots &
    {#1}_{mn} \end{pmatrix}}
\newcommand{\iprod}[2]{\left\langle {#1} , {#2} \right\rangle}
\newcommand{\norm}[1]{\left\Vert {#1} \right\Vert}
\newcommand{\abs}[1]{\left\vert {#1} \right\vert}
\renewcommand{\det}{\mathrm{det}}
\newcommand{\rank}{\mathrm{rank}}
\newcommand{\spn}{\mathrm{span}}
\newcommand{\row}{\mathrm{Row}}
\newcommand{\col}{\mathrm{Col}}
\renewcommand{\dim}{\mathrm{dim}}
\newcommand{\prefeq}{\succeq}
\newcommand{\pref}{\succ}
\newcommand{\seq}[1]{\{{#1}_n \}_{n=1}^\infty }
%\providecommand{\limp}{\underset{n \rightarrow \infty}{\overset{p}{\longrightarrow}}}
%\providecommand{\limp}{\underset{n \rightarrow \infty}{\overset{p}{\longrightarrow}}}
\providecommand{\limp}{\overset{p}{\longrightarrow}}
%\providecommand{\limd}{\underset{n \rightarrow \infty}{\overset{d}{\longrightarrow}}}
\providecommand{\limd}{\overset{d}{\longrightarrow}}
\newcommand{\sumobs}{\underset{i=1}{\overset{n}{\sum}}}
\newcommand{\prodobs}{\underset{i=1}{\overset{n}{\prod}}}
\renewcommand{\to}{{\rightarrow}}
\providecommand{\En}{\mathbb{E}_n}
\providecommand{\Gn}{\mathrm{G}_n}
\providecommand{\Var}{\mathrm{Var}}
\providecommand{\Vr}{\mathbb{V}}
\providecommand{\Er}{\mathbb{E}}
\providecommand{\Id}{\mathbf{I}}
\providecommand{\Ind}{\mathbf{1}}
\providecommand{\uvec}{\mathbf{1}}
\providecommand{\Rang}{\mathrm{Rang}}
\providecommand{\Trace}{\mathrm{Trace}}
\providecommand{\Tr}{\mathrm{Tr}}
\providecommand{\Cov}{\mathrm{Cov}}
\providecommand{\Diag}{\mathrm{Diag}}
\providecommand{\Pred}{\mathcal{P}}
\providecommand{\sp}{\mathrm{span}}
\providecommand{\CI}{\mathrm{CI}}
\providecommand{\reg}{\mathrm{r}}
\providecommand{\Likelihood}{\mathrm{L}}
\renewcommand{\Pr}{{\mathbb{P}}}
\providecommand{\set}[1]{\left\{#1\right\}}
\providecommand{\asyvar}{\mathrm{AsyVar}}
%\providecommand{\plim}{\operatornamewithlimits{plim}}
\newcommand\indep{\protect\mathpalette{\protect\independenT}{\perp}}
\def\independenT#1#2{\mathrel{\setbox0\hbox{$#1#2$}%
  \copy0\kern-\wd0\mkern4mu\box0}} 
%\renewcommand{\cite}{\citeasnoun}

%\DeclareMathOperator{\Trace}{Trace}
%\DeclareMathOperator{\Diag}{Diag}
%\DeclareMathOperator{\E}{E}
%\DeclareMathOperator{\En}{E_n}
%\DeclareMathOperator{\V}{V}
%\DeclareMathOperator{\I}{\mathbf{I}}
%\DeclareMathOperator{\Rang}{Rang}
%\DeclareMathOperator{\Cov}{Cov}
%\DeclareMathOperator{\Likelihood}{\mathcal{L}}
%\DeclareMathOperator{\Loglikelihood}{\log\mathcal{L}}
\makeatletter
\renewcommand{\@maketitle}{
  \null 
  \begin{center}%
    \rule{\linewidth}{1pt} 
     {\small \textsc{UGA M1: Econometrics 1}} \par 
    {\Large \textbf{\textsc{\@title}}} \par
    {\small \textsc{Michal Urdanivia,  Université de Grenoble Alpes,  Faculté d'\'Economie, GAEL}} \par
     {\small Courriel: \href{mailto:michal.wong-urdanivia@univ-grenoble-alpes.fr}{michal.wong-urdanivia@univ-grenoble-alpes.fr}} \par
    {\small \textsc{\@date}} \par
    \rule{\linewidth}{1pt} 
  \end{center}%
  \par \vskip 0.9em
}
\makeatother

\usepackage{fancyhdr}
\pagestyle{fancy}
\fancyhead[L]{Econometrics 1}
\fancyhead[R]{UGA M1, 2017-2018}
\fancyfoot[R]{\textcopyright \ \  Michal W. Urdanivia}
\usepackage{authblk}
%\usepackage{sectsty}
\usepackage{lipsum}

\theoremstyle{problem}
\newtheorem{problem}{Problem}

\newtheoremstyle{solution}% name
{2pt}% Space above
{12pt}% Space below
{}% Body font
{}% Indent amount
{\bfseries}% Theorem head font
{:}% Punctuation after theorem head
{.5em}% Space after theorem head
{}% Theorem head spec (can be left empty, meaning `normal')
\newtheorem{soln}{Solution}

\newenvironment{solution}
  {\begin{mdframed}\begin{soln}$\,$}
  {\end{soln}\end{mdframed}}
\title{Syllabus}
\date{\today}
\begin{document}
\maketitle
\section*{ Présentation du cours}
Ce cours est une introduction  à la théorie et pratique de l'économétrie. Une grande partie des thèmes abordés concernent le modèle de régression linéaire dans un cadre statique. Nous étudierons les méthodes d'estimation et d'inférence basés sur les moindres carrés ordinaires, les moindres carrés généralisés, la méthode des moments généralisée et les variables instrumentales, et le maximum de vraisemblance. Les résultats pour des échantillons de taille finie seront discutés, néanmoins, on s'intéressera principalement aux résultat asymptotiques.\\
Le public du cours est supposé familier des concepts de base d'algèbre linéaire, de calcul et d'analyse, et de statistique. A titre indicatif, vous pouvez consulter les ouvrages suivants:
\begin{itemize}
\item \emph{Matrix Differential Calculus with Applications in Statistics and Econometrics} par J. R. Magnus and H. Neudecker.
\item \emph{Statistical Inference} par  G. Casella and J. Berger.
\end{itemize}
Il y aura régulièrement du travail à préparer et à rendre(notamment
dans le cadre des séances de travaux dirigés) et ce travail comptera
pour l'évaluation au cours. Il s'agira aussi bien de travaux
analytiques, que de travaux pratiques sur données. Ce travail,
sera fait essentiellement sur Python et dans une moindre mesure sur R(pour lequels existe une pléthore de ressources en
ligne pour se former)\footnote{Il est important de noter que ce cours
  n'est pas un cours sur R ni sur Python, et ce faisant vous devrez vous autoformer à ce langage si vous souhaitez l'utiliser.}
\\
Le matériel pour le cours sera sur un site Github:\\
\url{ https://github. com/urdanivia/Courses/tree/master/UGA_L3Miash_Econometriie_1.}
\section*{Thèmes}
\begin{enumerate}
\item Le modèle de régression linéaire: moindres carrés ordinaires et propriétés pour des échantillons de taille finie, géométrie des moindres carrés, régression partitionnée, mesures de la qualité de l'ajustement.
\item Tests d'hypothèses et intervalles de confiance( pour des échantillons de taille finie).
\item \'Eléments de théorie asymptotique.
\item Modèle de régression linéaire dans un cadre asymptotique.
\item Moindres carrés généralisés.
\item Méthode des variables instrumentales.
\item Méthode des moments généralisés.
\item Modèles à équations simultanées.
\item Maximum de vraisemblance.
\end{enumerate}

\section*{Références}
Le cours ne suit pas un ouvrage de référence en particulier. Si vous le souhaitez vous  pouvez consulter:
\begin{itemize}
\item \emph{Advanced Econometrics}, par T. Amemiya.
\item \emph{Econometric Analysis of Cross Section and Panel Data}, par J. Wooldridge.
\item \emph{Econometric Theory and Methods}, par R. Davidson et  J. G. MacKinnon.
\end{itemize}
\end{document}
