%\documentclass[ignorenonframetext, compress, 9pt, xcolor=svgnames]{beamer} 
\input{../Config_slides}

\title[\'Econometrie 1: Introduction]{UGA L3 Miash: \'Econométrie 1\\ \textbf{Introduction}}
\date{\today}
\author{Michal W. Urdanivia\inst{*}}
\institute{\inst{*}Universit\'e de Grenoble Alpes, Facult\'e d'\'Economie, GAEL, \\e-mail: \href{mailto:michal.wong-urdanivia@univ-grenoble-alpes.fr}{michal.wong-urdanivia@univ-grenoble-alpes.fr}}


\begin{document}

\begin{frame}
\titlepage
\end{frame}

\begin{frame}
\frametitle{Contenu}
\tableofcontents[pausesections, pausesubsections]
\end{frame}

\section{ Présentation du cours}
\frame{\sectionpage}

\begin{frame}[allowframebreaks]{Objectifs et thèmes}
\begin{itemize}
\item Une introduction  à la théorie et pratique de l'économétrie.


\item Une partie importante concerne l'étude du  modèle de régression linéaire dans un cadre statique. 

\item On étudiera  les méthodes d'estimation et d'inférence relatives,

\begin{itemize}
\item à la méthode des moindres carrés ordinaires,
\item  aux moindres carrés généralisés,
\item  à la méthode des moments généralisée et aux variables instrumentales, 
\item au maximum de vraisemblance.
\end{itemize}

\item Le matériel pour ce cours(notes, corrections de tds, programmes,
  données) se trouvera sur un site GitHub dont je vous donnerai
  le lien via le Moodle UGA de ce cours avant la première séance de td.
\end{itemize}
\end{frame}

\begin{frame}[allowframebreaks]{Prérequis}
\begin{itemize}
\item Concepts de base d'algèbre linéaire, de calcul et d'analyse, et de statistique.

\item  A titre indicatif, vous pouvez consulter les ouvrages suivants(ou des
 manuels et/ou cours d'un niveau équivalent, notamment ceux relatifs à vos
 cours passés):
\begin{itemize}
\item \emph{Matrix Differential Calculus with Applications in Statistics and Econometrics} par J. R. Magnus and H. Neudecker.
\item \emph{Statistical Inference} par  G. Casella and J. Berger.
\end{itemize}

\item Notons que(dans la mesure où le temps nous le permet) nous ferons des
rappels lorsque nous en aurons besoin pour l'étude de certains points
du cours.
\end{itemize}
\end{frame}

\begin{frame}[allowframebreaks]{Organisation}
\begin{itemize}
\item Il y aura régulièrement du travail à préparer et à rendre(notamment
dans le cadre des séances de travaux dirigés) et ce travail comptera
pour l'évaluation au cours. 

\item Il s'agira aussi bien de travaux
analytiques, que de travaux pratiques sur données. 

\item Ce travail se fera essentiellement sur Python et dans
  une moindre mesure sur R.

\item \alert{Remarque:} ce cours n'est pas un cours de Python et vous
  devrez ce faisant vous auto-former.
\end{itemize}

\end{frame}


\section{L'\'Econométrie}
\frame{\sectionpage}

\begin{frame}[allowframebreaks]{Méthodes quantitatives pour l'analyse empirique}

Il s'agit notamment d'analyser:

\begin{itemize}
\item Des choix individuels: micro-économétrie.
\begin{itemize}
  \item Consommateurs: demande des biens marchands (dépenses
d'alimentation, de transport, ...), choix de marques, ...
  \item Firmes : investissement, main d'oeuvre, localisation, ...
  \item Salariés : durée de chômage, déterminants des salaires, ...
 \item  Ménages : épargne, portefeuille financier, ...
\end{itemize}
Et les données utilisées sont des données individuelles en coupe, ou en
panel (enquêtes répétées dans le temps).

\framebreak

\item Des relations entre des grands agrégats économiques :
  macro-économétrie
\begin{itemize}
  \item PNB, importations, consommation, épargne, taux d'intérêt, taux de
change, taux de salaire, ...
\item Déterminants de la croissance, du taux de chômage, ...
\end{itemize}
Dans ce cas les données sont souvent des séries temporelles sur un pays ou plusieurs pays, ...
\end{itemize}
\end{frame}

\begin{frame}[allowframebreaks]{Rôle de l'\'Economie}
Pour le traitement empirique d'une question donnée l'économètre peut
construire son modèle en s'appuyant implicitement ou explicitement
sur la théorie économique.

\medskip

Ces modèles peuvent être issus de(ou prendre des éléments dans) 

\begin{itemize}
\item la microéconomie: théorie du choix du consommateur, théorie du
  producteur, modèles de concurrence imparfaite, ...
\item la macroéconomie: modèles de croissance, commerce international,...
\end{itemize}

\framebreak

Notons que  la relation entre l'économétrie et l'économie peut
ne pas toujours apparaître comme évidente. 

\medskip

A partir d'une vue d'ensemble(et en simplifiant beaucoup), on peut
dire que cette relation concerne,

\begin{itemize}
  \item L'estimation de relations économiques
  \item Tester des théories économiques
  \item Prédire des variables d'intérêt économique
  \item L'évaluation de politiques économiques.
  \end{itemize}

\end{frame}


\begin{frame}[allowframebreaks]\frametitle{Rôle de la Statistique?}

  La théorie économique permet de construire des modèles qui
  caractérisent les relations entre des variables d'intérêt.

\medskip

  Les modèles économiques sont seulement des approximations du phénoméne étudié.
  
\medskip

Un modèle peut être suiffisamment complexe pour prendre en
    compte de nombreux facteurs, mais  il n'est pas possible d'exclure
    la possibilité que plusieurs facteurs ne soient pas pris en
    compte.

\medskip

C'est en ce sens qu'il convient de remplacer un modèle déterministe par un modèle probabiliste.
 

\framebreak

C'est dans ce cadre que l'économètre travaillera sur,

\begin{itemize}

\item L'estimation  des paramètres des modèles retenus pour
  l'analyse empirique d'un problème donné.
\item Les méthodes d'inférence pour juger de la qualité de son modèle.

\end{itemize}

\end{frame}

\section{Un exemple introductif}
\frame{\sectionpage}


\begin{frame}[allowframebreaks]\frametitle{Exemple: études et salaire}

On s'intéresse ici à la question de l'effet causal des études sur le
salaire.

\medskip

Il s'agit d'une question étudiée par de nombreux économistes. Voir
e.g., \cite{card2001estimating}.

\medskip

Soit $Y$ le salaire, $X$ le nombre d'années d'études, et $U$
une variable non observée par l'économètre et liée à $Y$.

\medskip

La première hypothèse que l'on peut faire est de supposer que la
relation entre $Y$, $X$ et $U$ est donnée par,

\begin{align*}
Y =& \alpha +  X\beta + U
\end{align*}

Où $\alpha$ et $\beta$ sont de paramètre inconnus que l'on veut
identifier à partir de nos observations(données) sur $(Y, X)$.

\medskip

Ces observations sont un échantillon de taille $N$ i.i.d, de $\{(Y_i, X_i)\}_{i=1}^N$.

\medskip

On veut identifier à partir des données $\beta$ qui mesure l'effet causal de $X$sur $Y$. 

\medskip

En effet, imaginons que l'on puisse fixer $U$ avec $U=\tilde{u}$ et
que l'on considère deux observations $i$, $j$, telles que $X_i = X_j +
\Delta X_i$.

\medskip

Nous avons alors,

\begin{align*}
Y_i - Y_j &= (X_j + \Delta X_i)\beta - X_j\beta\\
&= \Delta X_i \beta
\end{align*}

Et donc $\beta = \frac{\Delta Y_i}{\Delta X_i}$ mesure l'\emph{effet
  marginal} de $X$ sur $Y$ en \emph{contrôlant} les effets de
$U$.

\medskip

Malheuresement, en pratique $U$ n'est pas observé et nous devons
introduire des hypothèses pour pouvoir identifier $\beta$ à partir de
nos données.

\medskip

Imaginons que nous ne sachions calculer que des covariances. Nous
avons,

\begin{align*}
\Cov(Y, X) &= \Cov(X, X)\beta + \Cov(X, U)
\end{align*}

Nous pouvons identifier $\Cov(Y, X)$ et $\Cov(X, X) = \Var(X)$ dans
nos données(car $Y$ et $X$ sont observées) mais pas $\Cov(X, U)$(car
$U$ n'est pas observé).

\medskip

Il en résulte que pour identifier $\beta$ nous devons supposer que
$\Cov(X, U)= 0$.

\medskip

Et dans ce cas $\beta$ est identifié par,

\begin{align*}
\beta &= \frac{\Cov(Y, X)}{\Var(X)}
\end{align*}

Un estimateur de $\beta$ peut alors être construit en considérant les
contreparties empiriques de $\Cov(Y, X)$, et $\Var(X)$.

\medskip

En fait, il s'agit de l'estimateur des moindres carrés.

\medskip

L'hypothèse $\Cov(X, U)= 0$ est-elle pertinente dans cet exemple?

\medskip

En général non si l'on admet que $U$ contient des facteurs liés à $X$
qui ne sont pas pris en compte dans notre modèle, comme par exemple la
productivité des individus dans leurs études et emplois.

\medskip

Les méthodes que nous étudierons dans ce cours nous permettront de
traiter ce type de questions et d'autre encore.

\end{frame}


\begin{frame}[allowframebreaks]\frametitle{Références.}

\bibliographystyle{jpe}
\bibliography{../Biblio}

\end{frame}

\end{document}