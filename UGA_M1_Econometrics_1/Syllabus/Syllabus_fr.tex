\documentclass[12pt, reqno]{amsart}
\usepackage[svgnames]{xcolor}
\usepackage{amsmath}
\usepackage{amsthm}
\usepackage{amsfonts}
\usepackage{amssymb}
\usepackage{graphicx}
\usepackage{hyperref}
\usepackage[left=1in,right=1in,top=0.9in,bottom=0.9in]{geometry}
\usepackage{multirow}
\usepackage{verbatim}
\usepackage[greek,frenchb]{babel}
\usepackage[utf8]{inputenc}
\usepackage[T1]{fontenc}
\usepackage{pdfpages}
\usepackage{enumitem}
\usepackage{natbib}
\usepackage{ctable}
\usepackage{lscape}
\usepackage{array}
\usepackage{bbm}
\usepackage{hyperref}
\usepackage{graphicx}
\usepackage{epsf}
\usepackage{fancyhdr}
\usepackage{setspace}
\usepackage{verbatim}
\usepackage{courier}
\usepackage{color}
\usepackage[normalem]{ulem}
\usepackage{hyperref}
\usepackage{fncylab}
\usepackage{authblk}
\usepackage{textcomp}
%\usepackage{sectsty}
\usepackage{lipsum}
\hypersetup{
  colorlinks   = true, %Colours links instead of ugly boxes
  urlcolor     = blue, %Colour for external hyperlinks
  linkcolor    = DarkBlue, %Colour of internal links
  citecolor   = DarkBlue %Colour of citations
}
\newtheorem{theoreme}{Théorème}
\newtheorem{conjecture}{Conjecture}[section]
\newtheorem{corollary}{Corollary}[section]
\newtheorem{lemme}{Lemme}[section]
\newtheorem{proposition}{Proposition}[section]
\newtheorem{assumption}{}
\labelformat{assumption}{H\arabic{assumption}}
\renewcommand{\theassumption}{Hypothèse H\arabic{assumption}}
\newtheorem{remarque}{Remarque}
\newtheorem{definition}{}
\labelformat{definition}{D\arabic{definition}}
\newtheorem{exemple}{Exemple}
\renewcommand{\thedefinition}{Définition D\arabic{definition}}
\newcommand{\argmax}{\operatornamewithlimits{arg\,max\,}}
\newcommand{\argmin}{\operatornamewithlimits{arg\,min\,}}
\providecommand{\plim}{\operatornamewithlimits{plim}}
\def\inprobLOW{\rightarrow_p}
\def\inprobHIGH{\,{\buildrel p \over \rightarrow}\,} 
\def\inprob{\,{\inprobHIGH}\,} 
\def\indist{\,{\buildrel d \over \rightarrow}\,} 
\def\F{\mathbb{F}}
\def\R{\mathbb{R}}
\newcommand{\gmatrix}[1]{\begin{pmatrix} {#1}_{11} & \cdots &
    {#1}_{1n} \\ \vdots & \ddots & \vdots \\ {#1}_{m1} & \cdots &
    {#1}_{mn} \end{pmatrix}}
\newcommand{\iprod}[2]{\left\langle {#1} , {#2} \right\rangle}
\newcommand{\norm}[1]{\left\Vert {#1} \right\Vert}
\newcommand{\abs}[1]{\left\vert {#1} \right\vert}
\renewcommand{\det}{\mathrm{det}}
\newcommand{\rank}{\mathrm{rank}}
\newcommand{\spn}{\mathrm{span}}
\newcommand{\row}{\mathrm{Row}}
\newcommand{\col}{\mathrm{Col}}
\renewcommand{\dim}{\mathrm{dim}}
\newcommand{\prefeq}{\succeq}
\newcommand{\pref}{\succ}
\newcommand{\seq}[1]{\{{#1}_n \}_{n=1}^\infty }
%\providecommand{\limp}{\underset{n \rightarrow \infty}{\overset{p}{\longrightarrow}}}
%\providecommand{\limp}{\underset{n \rightarrow \infty}{\overset{p}{\longrightarrow}}}
\providecommand{\limp}{\overset{p}{\longrightarrow}}
%\providecommand{\limd}{\underset{n \rightarrow \infty}{\overset{d}{\longrightarrow}}}
\providecommand{\limd}{\overset{d}{\longrightarrow}}
\newcommand{\sumobs}{\underset{i=1}{\overset{n}{\sum}}}
\newcommand{\prodobs}{\underset{i=1}{\overset{n}{\prod}}}
\renewcommand{\to}{{\rightarrow}}
\providecommand{\En}{\mathbb{E}_n}
\providecommand{\Gn}{\mathrm{G}_n}
\providecommand{\Var}{\mathrm{Var}}
\providecommand{\Vr}{\mathbb{V}}
\providecommand{\Er}{\mathbb{E}}
\providecommand{\Id}{\mathbf{I}}
\providecommand{\Ind}{\mathbf{1}}
\providecommand{\uvec}{\mathbf{1}}
\providecommand{\Rang}{\mathrm{Rang}}
\providecommand{\Trace}{\mathrm{Trace}}
\providecommand{\Tr}{\mathrm{Tr}}
\providecommand{\Cov}{\mathrm{Cov}}
\providecommand{\Diag}{\mathrm{Diag}}
\providecommand{\Pred}{\mathcal{P}}
\providecommand{\sp}{\mathrm{span}}
\providecommand{\CI}{\mathrm{CI}}
\providecommand{\reg}{\mathrm{r}}
\providecommand{\Likelihood}{\mathrm{L}}
\renewcommand{\Pr}{{\mathbb{P}}}
\providecommand{\set}[1]{\left\{#1\right\}}
\providecommand{\asyvar}{\mathrm{AsyVar}}
%\providecommand{\plim}{\operatornamewithlimits{plim}}
\newcommand\indep{\protect\mathpalette{\protect\independenT}{\perp}}
\def\independenT#1#2{\mathrel{\setbox0\hbox{$#1#2$}%
  \copy0\kern-\wd0\mkern4mu\box0}} 
%\renewcommand{\cite}{\citeasnoun}

%\DeclareMathOperator{\Trace}{Trace}
%\DeclareMathOperator{\Diag}{Diag}
%\DeclareMathOperator{\E}{E}
%\DeclareMathOperator{\En}{E_n}
%\DeclareMathOperator{\V}{V}
%\DeclareMathOperator{\I}{\mathbf{I}}
%\DeclareMathOperator{\Rang}{Rang}
%\DeclareMathOperator{\Cov}{Cov}
%\DeclareMathOperator{\Likelihood}{\mathcal{L}}
%\DeclareMathOperator{\Loglikelihood}{\log\mathcal{L}}
\makeatletter
\renewcommand{\@maketitle}{
  \null 
  \begin{center}%
    \rule{\linewidth}{1pt} 
     {\small \textsc{UGA M1: \'Econométrie 1}} \par 
    {\Large \textbf{\textsc{\@title}}} \par
    {\small \textsc{Michal Urdanivia,  Université de Grenoble Alpes,  Faculté d'\'Economie, GAEL}} \par
     {\small Courriel: \href{mailto:michal.wong-urdanivia@univ-grenoble-alpes.fr}{michal.wong-urdanivia@univ-grenoble-alpes.fr}} \par
    {\small \textsc{\@date}} \par
    \rule{\linewidth}{1pt} 
  \end{center}%
  \par \vskip 0.9em
}
\makeatother

\usepackage{fancyhdr}
\pagestyle{fancy}
\fancyhead[L]{\'Econométrie 1}
\fancyhead[R]{UGA M1, 2017-2018}
\fancyfoot[R]{\textcopyright \ \  Michal W. Urdanivia}
\usepackage{authblk}
%\usepackage{sectsty}
\usepackage{lipsum}
\title{Syllabus}
\date{\today}
\begin{document}
\maketitle
\section*{ Présentation du cours}
Ce cours est une introduction  à la théorie et pratique de l'économétrie. Le principal thème du cours est le modèle de régression linéaire, son estimation et les méthodes d'inférence. D'autres thèmes concernent le traitement de l'hétéroscédasticité et de l'endogénéité par la méthodes des variables instrumentales dans le modèle de régression linéaire.\\
Les prérequis du cours sont  des cours d'algèbre linéaire, de calcul et d'analyse, et de statistique d'un niveau équivalent à,
\begin{itemize}
\item \emph{Matrix Differential Calculus with Applications in Statistics and Econometrics} par J. R. Magnus and H. Neudecker.
\item \emph{Statistical Inference} par  G. Casella and J. Berger.
\end{itemize}
Outre des exercices analytiques, il y aura des travaux d'application(empirique) des méthodes vues en cours(notamment dans les séances de TD). Les correction se feront essentiellement sur Python(et dans une moindre mesure sur R)\footnote{Notez que ce cours n'est pas un cours nin de Python ni de R et au-delà de ce qui se fera dans les corrections. Vous aurez donc à vous former sur un certains nombre de points de façon autonome.}\\
Le matériel pour le cours(notes de lecture, devoirs, données) seront ici:
\url{https://github.com/urdanivia/Courses/tree/master/UGA_M1_Econometrics_1}.


\section*{Thèmes}
\begin{enumerate}
\item Introduction.
\begin{itemize}
  \item \cite{w2013} chapter 1, \cite{ap2014} introduction,
    \cite{sw2009} chapter 1,
    \cite{burtless1985}
  \end{itemize}
\item Rappels de Probabilité.
 \begin{itemize}
  \item \cite{w2013} appendix B, \cite{sw2009} chapter 2, \cite{menzel2009}
  \end{itemize}
  \item Rappels de Statistique.
   \begin{itemize}
  \item \cite{w2013} appendix C, \cite{sw2009} chapter 3, \cite{menzel2009},
    \cite{woodbury1987}, \cite{ap2014} chapter 1
  \end{itemize}
\item Regression linéaire simple.
  \begin{itemize}
  \item \cite{w2013} chapter 2, \cite{sw2009} chapter 4-5, 17, \cite{ap2014} chapter 2
  \end{itemize}
\item Regression linéaire multiple.
  \begin{itemize}
  \item \cite{w2013} chapters 3-7, \cite{sw2009} chapter 6-9, 18, \cite{ap2014} chapter 2, \cite{krueger1993},
    \cite{dinardo1997}, \cite{dale2002}, \cite{card1997myth}
  \end{itemize}
\item Hétéroscédasticité. 
  \begin{itemize}
  \item \cite{w2013} chapter 8
  \end{itemize}
\item Variables instrumentales.
  \begin{itemize}
  \item \cite{w2013} chapter 15,  \cite{sw2009} chapter 12, \cite{ap2014} chapter 3, \cite{angrist1990},
    \cite{angrist1991}, \cite{ashenfelter1994}
  \end{itemize}
\end{enumerate}

\section*{Lectures et références}

Les notes relatives au cours(aux lectures faites des thèmes) seront mises à votre disposition sur le site du cours.
Elles ne suivent pas un ouvrage de référence mais vous pouvez les complèter avec  notamment:
\begin{itemize}
\item Angrist and Pischke's {\slshape
  Mastering 'Metrics}.  
\item Wooldridge's {\slshape Introductory Econometrics: A Modern Approach}
is a more typical textbook. The course notes were originally based on
this book. The most recent edition is available in the UBC
bookstore. An older edition would also be fine. 
\item Stock \& Watson's {\slshape Introduction to Econometrics} is another
typical textbook. It is slightly less technical and more readable than
Wooldridge. 
\end{itemize}
Remarquez que les notes mises à votre disposition seront en Anglais. Cela résulte de ce que les principales références sont elles aussi en Anglais.\\
Enfin, parmi les références vous trouverez aussi un certain nombre d'articles qui nous permettront d'illustrer certains points du cours. Certains de ces articles feront l'objet d'une présentation de votre part.




\bibliographystyle{jpe}
\bibliography{../Biblio.bib}
\end{document}
