%\documentclass[12pt, reqno]{amsart}
%\documentclass[12pt, reqno, fleqn]{amsart}
\documentclass[12pt, reqno]{amsart}
\usepackage[svgnames]{xcolor}
\usepackage{amsmath}
\usepackage{amsthm}
\usepackage{amsfonts}
\usepackage{amssymb}
\usepackage{graphicx}
\usepackage{hyperref}
\usepackage[left=1in,right=1in,top=0.9in,bottom=0.9in]{geometry}
\usepackage{multirow}
\usepackage{verbatim}
\usepackage[greek,frenchb]{babel}
\usepackage[utf8]{inputenc}
\usepackage[T1]{fontenc}
\usepackage{pdfpages}
\usepackage{enumitem}
\usepackage{natbib}
\usepackage{ctable}
\usepackage{lscape}
\usepackage{array}
\usepackage{bbm}
\usepackage{hyperref}
\usepackage{graphicx}
\usepackage{epsf}
\usepackage{fancyhdr}
\usepackage{setspace}
\usepackage{verbatim}
\usepackage{courier}
\usepackage{color}
\usepackage[normalem]{ulem}
\usepackage{hyperref}
\usepackage{fncylab}
\usepackage{authblk}
\usepackage{textcomp}
%\usepackage{sectsty}
\usepackage{lipsum}
\hypersetup{
  colorlinks   = true, %Colours links instead of ugly boxes
  urlcolor     = blue, %Colour for external hyperlinks
  linkcolor    = DarkBlue, %Colour of internal links
  citecolor   = DarkBlue %Colour of citations
}
\newtheorem{theoreme}{Théorème}
\newtheorem{conjecture}{Conjecture}[section]
\newtheorem{corollary}{Corollary}[section]
\newtheorem{lemme}{Lemme}[section]
\newtheorem{proposition}{Proposition}[section]
\newtheorem{assumption}{}
\labelformat{assumption}{H\arabic{assumption}}
\renewcommand{\theassumption}{Hypothèse H\arabic{assumption}}
\newtheorem{remarque}{Remarque}
\newtheorem{definition}{}
\labelformat{definition}{D\arabic{definition}}
\newtheorem{exemple}{Exemple}
\renewcommand{\thedefinition}{Définition D\arabic{definition}}
\newcommand{\argmax}{\operatornamewithlimits{arg\,max\,}}
\newcommand{\argmin}{\operatornamewithlimits{arg\,min\,}}
\providecommand{\plim}{\operatornamewithlimits{plim}}
\def\inprobLOW{\rightarrow_p}
\def\inprobHIGH{\,{\buildrel p \over \rightarrow}\,} 
\def\inprob{\,{\inprobHIGH}\,} 
\def\indist{\,{\buildrel d \over \rightarrow}\,} 
\def\F{\mathbb{F}}
\def\R{\mathbb{R}}
\newcommand{\gmatrix}[1]{\begin{pmatrix} {#1}_{11} & \cdots &
    {#1}_{1n} \\ \vdots & \ddots & \vdots \\ {#1}_{m1} & \cdots &
    {#1}_{mn} \end{pmatrix}}
\newcommand{\iprod}[2]{\left\langle {#1} , {#2} \right\rangle}
\newcommand{\norm}[1]{\left\Vert {#1} \right\Vert}
\newcommand{\abs}[1]{\left\vert {#1} \right\vert}
\renewcommand{\det}{\mathrm{det}}
\newcommand{\rank}{\mathrm{rank}}
\newcommand{\spn}{\mathrm{span}}
\newcommand{\row}{\mathrm{Row}}
\newcommand{\col}{\mathrm{Col}}
\renewcommand{\dim}{\mathrm{dim}}
\newcommand{\prefeq}{\succeq}
\newcommand{\pref}{\succ}
\newcommand{\seq}[1]{\{{#1}_n \}_{n=1}^\infty }
%\providecommand{\limp}{\underset{n \rightarrow \infty}{\overset{p}{\longrightarrow}}}
%\providecommand{\limp}{\underset{n \rightarrow \infty}{\overset{p}{\longrightarrow}}}
\providecommand{\limp}{\overset{p}{\longrightarrow}}
%\providecommand{\limd}{\underset{n \rightarrow \infty}{\overset{d}{\longrightarrow}}}
\providecommand{\limd}{\overset{d}{\longrightarrow}}
\newcommand{\sumobs}{\underset{i=1}{\overset{n}{\sum}}}
\newcommand{\prodobs}{\underset{i=1}{\overset{n}{\prod}}}
\renewcommand{\to}{{\rightarrow}}
\providecommand{\En}{\mathbb{E}_n}
\providecommand{\Gn}{\mathrm{G}_n}
\providecommand{\Var}{\mathrm{Var}}
\providecommand{\Vr}{\mathbb{V}}
\providecommand{\Er}{\mathbb{E}}
\providecommand{\Id}{\mathbf{I}}
\providecommand{\Ind}{\mathbf{1}}
\providecommand{\uvec}{\mathbf{1}}
\providecommand{\Rang}{\mathrm{Rang}}
\providecommand{\Trace}{\mathrm{Trace}}
\providecommand{\Tr}{\mathrm{Tr}}
\providecommand{\Cov}{\mathrm{Cov}}
\providecommand{\Diag}{\mathrm{Diag}}
\providecommand{\Pred}{\mathcal{P}}
\providecommand{\sp}{\mathrm{span}}
\providecommand{\CI}{\mathrm{CI}}
\providecommand{\reg}{\mathrm{r}}
\providecommand{\Likelihood}{\mathrm{L}}
\renewcommand{\Pr}{{\mathbb{P}}}
\providecommand{\set}[1]{\left\{#1\right\}}
\providecommand{\asyvar}{\mathrm{AsyVar}}
%\providecommand{\plim}{\operatornamewithlimits{plim}}
\newcommand\indep{\protect\mathpalette{\protect\independenT}{\perp}}
\def\independenT#1#2{\mathrel{\setbox0\hbox{$#1#2$}%
  \copy0\kern-\wd0\mkern4mu\box0}} 
%\renewcommand{\cite}{\citeasnoun}

%\DeclareMathOperator{\Trace}{Trace}
%\DeclareMathOperator{\Diag}{Diag}
%\DeclareMathOperator{\E}{E}
%\DeclareMathOperator{\En}{E_n}
%\DeclareMathOperator{\V}{V}
%\DeclareMathOperator{\I}{\mathbf{I}}
%\DeclareMathOperator{\Rang}{Rang}
%\DeclareMathOperator{\Cov}{Cov}
%\DeclareMathOperator{\Likelihood}{\mathcal{L}}
%\DeclareMathOperator{\Loglikelihood}{\log\mathcal{L}}
\makeatletter
\renewcommand{\@maketitle}{
  \null 
  \begin{center}%
    \rule{\linewidth}{1pt} 
     {\small \textsc{UGA M1: \'Econométrie 1}} \par 
    {\Large \textbf{\textsc{\@title}}} \par
    {\small \textsc{Michal Urdanivia,  Université de Grenoble Alpes,  Faculté d'\'Economie, GAEL}} \par
     {\small Courriel: \href{mailto:michal.wong-urdanivia@univ-grenoble-alpes.fr}{michal.wong-urdanivia@univ-grenoble-alpes.fr}} \par
    {\small \textsc{\@date}} \par
    \rule{\linewidth}{1pt} 
  \end{center}%
  \par \vskip 0.9em
}
\makeatother

\usepackage{fancyhdr}
\pagestyle{fancy}
\fancyhead[L]{\'Econométrie 1}
\fancyhead[R]{UGA M1, 2017-2018}
\fancyfoot[R]{\textcopyright \ \  Michal W. Urdanivia}
\usepackage{authblk}
%\usepackage{sectsty}
\usepackage{lipsum}
\title{Problème 1: régression linéaire simple}
\date{\today}
\begin{document}
\maketitle
\section*{Exercice 1}
Les données "BWGHT.dta" contiennent des informations sur les
naissances par des femmes au \'Etats-Unis. Deux variables retiennent
ici notre attention: le poids à la naissance des nouveaux-nés(mesuré
en onces(variable $bwght$), et le nombre moyen de cigarettes
fumées par jour durant la période de grosesse(variable
$cigs$). On souhaite estimer la régression suivante 
sur les données relatives à $N = 1388$ naissances,

\[bwght_i  = \beta_0 + \beta_1 cigs_i + U_i, \ \ \ i = 1, ...,N\]

où l'on suppose que $\Er(U_i| cigs_i) = 0$. L'échantillon $\{(bwght_i,
cigs_i)\}_{i=1}^N$ est par ailleurs supposé i.i.d.

\begin{enumerate}
\item Donnez une interprétation de $\beta_0$ et de $\beta_1$.
\item L'estimation par moindres carrés donne $\hat{\beta_0} = 119.77$
  et $\hat{\beta_1} = -0.514$. Rappelez les formules pour obtenir ces estimateurs.
\item Quel est le poids prédit par ce modèle quand $cigs = 0$, et
  celui quand $cigs = 20$? Commentez la différence.
\item L'hypothèse $\Er(U_i| cigs_i) = 0$ vous semble t-elle
  justifiée?(argumentez votre réponse).
\item Présentez un script (ou "notebook jupyter") Python permettant de
  calculer $\hat{\beta_1}$ et $\hat{\beta_1}$(avec lequel vous devez
  trouver les valeurs indiquées plus-haut).
\end{enumerate}

\section*{Exercice 2}
Le site de Joshua Angrist contient les données utilisées dans
plusieurs de ses travaux\footnote{Allez à
  \url{(http://economics. mit.edu/faculty/angrist/data1/data/anglavy99)}}. On
s'intéresse ici à celle relatives à \cite{angrist1999}. Dans ce
travail les auteurs s'intéressent à l'effet causal de la taille des classes
sur les résultats des élèves à des tests d'évaluation. Pour identifier
cet effet causal ils considèrent le cas des écoles israéliennes dont
une des règles de fonctionnement leur permet d'utiliser une stratégie
par variables instrumentales. Comme cette méthode sera vue dans une
partie plus avancée du cours, on se contentera pour le moment des
aspects du travail qui emploient des régressions "de base"

\begin{enumerate}
\item Lisez l'article jusqu'à la fin de la section 1(au moins), et
  faites un résumé de deux pages au plus en présentant la
  problèmatique, et les données.
\item Téléchargez les données et calculez les statistiques
  descriptives du tableau 1 pour l'échantillon de la cinquième
  année("5th grade", soit 2019 observations, voir page 539 dans la revue, 7
  dans le fichier pdf). Vous fournirez le code Python utilisé. Notez
  que l'unité d'observation est la classe, que le nombre
  d'inscrits( "enrollment" en anglais dans l'article)
  est appelé $c\_size$, que le pourcentage d'élèves
  désavantagés("percent disadvantaged" en anglais dans l'article) est
  appelé $tipuach$.\\
En outre, pour reproduire exactement les nombres du tableau 4 vous
devez suivre les indications de la note de bas de page 11 et
restreindre l'échantillon aux écoles avec au moins 5 inscriptions et
des classes de taille inférieures à 45. Il y a aussi certaines
corrections peu évidentes à faire sur les données. Ainsi il y a des
 des observations qui présentent une moyenne au
test de maths(variable $avgmath$) et une moyenne au test en expression
écrite $avgberb$ supérieures à 100, en raison d'erreurs dans
l'enregistrement des données. Les valeurs correctes sont 87.606 et
81.246 (et non 187.606 et 181.246). Enfin, il y a un score en maths
non manquant pour une observation avec $mathsize = 0$ ce qui est
impossible(car pas d'élèves en maths). Il faut ici remplacer cette
valeur non manquante par une valeur manquante. Le code qui accopagne
ce TD règle les problèmes ci-dessus mentionnés.
\item Un débat courant entre économistes, et d'autres chercheurs en
  sciences sociales concerne l'arbitrage entre les coûts et les
  bénéfices associés à la réduction de la taille des classes. Quel
  serait les signe de l'effet de la réduction de la taille des classes
  sur les résultats scolaires(tels que mesurés par des tests comme
  ceux de l'article) si cette réduction valait la peine?
\item Estimez les régression des résultats en maths sur la taille des
  classes(et une constante), et des résultats en expression écrite sur la taille des
  classes(et une constante. Quel est le signe de ces relations?
  Sont-elles significativement différentes de zéro?
\end{enumerate}
\bibliographystyle{jpe}
\bibliography{../../Biblio}


\end{document}

